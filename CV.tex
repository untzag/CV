% document
\documentclass{resume}  % Use the custom resume.cls style

% header information
\name{Blaise J Thompson}
\address{725 W Washington Ave. Apt. 306, Madison, WI 53715}
\address{1$\cdot$424$\cdot$225$\cdot$2493 \\ blaise@untzag.com \\
  \href{http://blaise.social/}{blaise.social}}
\address{\today}

\begin{document}

\begin{rSection}{Education}
  \begin{rSubsection}{University of Wisconsin-Madison}{2011 - Present}
    {PhD, Analytical Chemistry}{Madison WI}
    \item Researcher in John C. Wright group.
    \item Focus on ultrafast materials spectroscopy and instrument development.
  \end{rSubsection}
  \begin{rSubsection}{Bates College}{2007 - 2011}{BS, Chemistry; Minor, Philosophy; Concentration:
      Applying Mathematical Methods}{Lewiston ME}
    \item Senior thesis completed in lab of Matthew J. Cote: \\ \textit{Investigations of
      Plasmon Polaritions with Total Internal Reflection \& Atomic Force Microscopy}.
  \end{rSubsection}
\end{rSection}

\begin{rSection}{Experience}
  \begin{rSubsection}{John C. Wright Group - ultrafast materials spectroscopy}{2011 - Present}
    {Graduate Assistant}{Madison WI}
    \item Designed and constructed software tools to collect and process multidimensional spectra
    \item Developed novel tools to streamline OPA tuning procedures.
    \item Worked in collaboration with Physical and Materials chemists to address challenges in
      solar energy generation.
  \end{rSubsection}
  \begin{rSubsection}{Matthew J. Cote Group - microscopy and plasmonics}{2009 - 2011}
    {Undergraduate Researcher}{Lewiston ME}
    \item Contiguous work for two academic years and intervening summer.
    \item Designed and constructed a combined total internal reflection / atomic force microscope.
    \item Worked independently and in groups leading other students.
    \item Coordinated work with my advisor and other staff and faculty.
    \item Wrote a comprehensive thesis on my work.
  \end{rSubsection}
  \begin{rSubsection}{Michael Dailey Group - neuroscience}{2008}{Undergraduate Researcher}
    {Iowa City IA}
    \item Dissected and prepared mouse brain samples for in-vivo microglial imaging studies.
    \item Trained to utilize confocal microscopy setup.
  \end{rSubsection}
  \begin{rSubsection}{Peter L. Nagy Group - epigenetics}{2007}
    {High School Researcher}{Iowa City IA}
    \item Designed and created plasmid, teaching myself from reference materials.
    \item Inserted plasmid into yeast.
  \end{rSubsection}
\end{rSection}

\clearpage

\begin{rSection}{Projects (see my \href{https://github.com/untzag}{GitHub})}
  \begin{rSubsection}{Creator: \href{http://wright.tools}{WrightTools}}{2014 - Present}
    {Tools for loading, processing, and plotting multidimensional spectroscopy data.}{}
    \item Single processing toolkit for wide variety of instrumental data, built to be extensible
      as more data-types become relevant
    \item Offers specialized interactions, such as transformations, that are particularly suited to
      multidimensional spectroscopy
    \item Online documentation through Sphinx and ReadTheDocs
    \item Project managed with several graduate student and undergraduate contributors, active
      issue tracking, version control and an extensive testing suite
    \item Central package used as a data management pipeline by other packages simulating and
      acquiring multidimensional spectra
  \end{rSubsection}
  \begin{rSubsection}{Creator: \href{https://github.com/wright-group/PyCMDS}{PyCMDS}}
    {2015 - Present}
    {Unified software for controlling hardware and collecting data in the Wright group.}{}
    \item Supplies modular hardware control, calibration, and orchestration during complex,
      long-lasting CMDS experiments.
    \item Provides interface to optomechanical hardware from a variety of manufacturers, including
      National Instruments, Thorlabs, Horiba, Newport, and Aerotech. Also controls hardware built
      and customized in-house.
    \item Delights users with advanced features such as automatic data backup and notification via
      Slack.
    \item In conjunction with contemporaneous hardware improvements, algorithmic improvements in
      acquisition strategy \textbf{increased scan rate by up to two orders of magnitude} over
      previous software.
  \end{rSubsection}
  \begin{rSubsection}{Creator: \href{https://github.com/wright-group/FilterWheels}{automated filter
        wheel assembly}}{2017}{Custom filter wheel hardware, electronics and firmware.}{}
    \item Allows for new experimental degrees of freedom within the Wright group.
    \item Designed and constructed custom chases in collaboration with the department machine shop.
    \item Designed custom circuit board using KiCad, ordered supplies from appropriate online
      retailers
    \item Designed and implemented serial interface and Arduino firmware, including semi-syncronus
      motion, low-level C++ string processing, and microstepping control for enhanced acquisition
      time efficiency
  \end{rSubsection}
  \begin{rSubsection}{Creator: \href{https://github.com/untzag/tidy_headers}{tidy\_headers}}{2017}
    {Rapidly write data from python to plain text, and back again.}{}
    \item Dependency of larger projects like WrightTools, and used directly for custom
      applications.
  \end{rSubsection}
  \begin{rSubsection}{Founder: \href{https://github.com/wright-group/WrightSim}{WrightSim}}{2017 -
      Present}
    {Efficient, flexable simulation package for multidimensional spectroscopy.}{}
    \item Uses Liouville's theorem to numerically simulate nonlinear spectroscopy.
    \item I was also a principle contributor to the predecessor of WrightSim,
      \href{https://github.com/wright-group/NISE}{NISE}.
  \end{rSubsection}
  \begin{rSubsection}{Contributor: InGaAs array}{2015 - 2016}
    {Quickly and cheaply acquire near-infrared pulse spectra.}{}
    \item Purchased bare array and specialty ADC chip from Hamamatsu, without purchasing expensive
      control and timing box. Designed and produced custom circuit board and enclosure for device.
    \item Wrote firmware to handle serial communication between ADC, acquisition software.
    \item Used advanced features such as watchdog timers to handle unexpected timing and
      communication problems.
  \end{rSubsection}
  \begin{rSubsection}{Contributor: \href{https://github.com/dib-lab/osf-cli}{osfclient}}{2017}
    {A python library and command-line client for file storage on OSF}{}
    \item Added Windows functionality, assisted in various debugging efforts in early version of
      osfcli
  \end{rSubsection}
\end{rSection}

\clearpage

\begin{rSection}{Publications}
  \begin{etaremune}[leftmargin = 1.75em]
    \item Horak, Erik H., Rea, Morgan T. Heylman, Kevin D. Gelbwaser-Klimovsky, David, Saikin,
      Semion K., \textbf{Thompson, Blaise J.}, Kohler, Daniel D., Knapper, Kassandra A., Wei, Wei;
      Pan, Feng; Gopalan, Padama; Wright, John C.; Aspuru-Guzik, Alan; \& Goldsmith, Randall H.
      (2018).
      Exploring Electronic Structure and Order i Polymers via Single-Particle Microresonator
      Spectroscopy.
      \textit{Nano Letters}, in press
    \item Kohler, D. D., \textbf{Thompson, B. J.}, \& Wright, J. C. (2017). Frequency-domain
      coherent multidimensional spectroscopy when dephasing rivals pulsewidth:
      Disentangling material and instrument response.
      \textit{The Journal of Chemical Physics}, 147(8), 84202.
      \href{https://doi.org/10.1063/1.4986069}{doi:10.1063/1.4986069}
    \item Czech, K. J., \textbf{Thompson, B. J.}, Kain, S., Ding, Q., Shearer, M. J., Hamers,
      R. J., Jin, S., \& Wright, J. C. (2015). Measurement of Ultrafast Excitonic Dynamics of
      Few-Layer MoS$_2$ Using State-Selective Coherent Multidimensional Spectroscopy.
      \textit{ACS Nano}, 9(12), 12146–12157.
      \href{https://doi.org/10.1021/acsnano.5b05198}{doi:10.1021/acsnano.5b05198}
    \item Fu, Y., Meng, F., Rowley, M. B., \textbf{Thompson, B. J.}, Shearer, M. J.,
      Ma, D., Hamers, R. J., Wright J., \& Jin, S. (2015). Solution Growth of Single Crystal
      Methylammonium Lead Halide Perovskite Nanostructures for Optoelectronic and
      Photovoltaic Applications.
      \textit{Journal of the American Chemical Society}, 137(17), 5810–5818.
      \href{https://doi.org/10.1021/jacs.5b02651}{doi:10.1021/jacs.5b02651}
    \item Cabán-Acevedo, M., Kaiser, N. S., English, C. R., Liang, D., \textbf{Thompson, B. J.},
      Chen, H.-E., Czech, K. C., Wright, J. C., Hamers, R. J., \& Jin, S. (2014).
      Ionization of High-Density Deep Donor Defect States Explains the Low
      Photovoltage of Iron Pyrite Single Crystals.
      \textit{Journal of the American Chemical Society}, 136(49), 17163–17179.
      \href{https://doi.org/10.1021/ja509142w}{10.1021/ja509142w}
  \end{etaremune}
\end{rSection}

\begin{rSection}{Presentations}
  \begin{etaremune}[leftmargin = 1.75 em]
    \item Presentation, Chaos and Complexity Seminar:
      ``Nonlinear Multidimensional Spectroscopy''
      2017. Madison, WI USA
      \href{https://github.com/untzag/CV/raw/master/supplementary/2017-11-07\%20chaos.pdf}{PDF}
    \item Poster, Coherent Multidimensional Spectroscopy'':
      ``A Robust, Fully Automated Algorithm to Collect High Quality OPA Tuning Curves''
      2016. Groningen, the Netherlands
      \href{https://github.com/untzag/CV/raw/master/supplementary/2016-06-01\%20CMDS.pdf}{PDF}
    \item Poster, Midwest Universities Analytical Chemistry Conference'':
      ``Utilizing Coherent Multidimensional Spectroscopy to Investigate Nanomaterials for Solar Energy
      Generation.''
      2012. Madison, WI USA
    \item Poster, Mount David Summit'':
      ``Spectroscopic Investigation of Plasmonic Nanoparticles.''
      2011. Bates College; Lewiston, ME USA
  \end{etaremune}
\end{rSection}

\begin{rSection}{Awards \& Honors}
  \begin{rSubsection}{Roger Carlson Award}{2017}{}{}
	  \item Awarded by the University of Wisconsin Chemistry department for excellence in research.
  \end{rSubsection}
  \begin{rSubsection}{Taylor Teaching Award}{2016}{}{}
    \item  Selected by University of Wisconsin Chemistry students and Faculty as one of the most
      outstanding Teaching Assistants of the 2015-2016 School Year.
  \end{rSubsection}
  \begin{rSubsection}{Rodney F. Johonnot Graduate Award}{2011}{}{}
    \item  Selected by Bates Faculty as most deserving of aid in furthering his or her studies in
      professional or postgraduate work.
  \end{rSubsection}
  \begin{rSubsection}{Bates College Key}{2011}{}{}
    \item Awarded by Bates Faculty and staff to 20 students in each graduating class based on
      academic standing, character, campus and community service, leadership, and future promise.
  \end{rSubsection}
\end{rSection}

\clearpage

\begin{rSection}{Teaching Experience}
  \begin{rSubsection}{Fundamentals of Analytical Science}{2018}{Teaching Assistant, 1 semester}
    {UW-Madison}
    \item Led laboratory and discussion sections.
  \end{rSubsection}
  \begin{rSubsection}{Graduate Chemical Instrumentation: Design \& Control (Electronics)}{2017}
    {Teaching Assistant, 1 semester}{UW-Madison}
    \item Led laboratory section of course.
    \item Assisted students during extended independent instrument design and construction.
  \end{rSubsection}
  \begin{rSubsection}{Graduate Instrumental Analysis}{2012, 2015}{Teaching Assistant, 2
      semesters}{UW-Madison}
    \item Led laboratory section of course.
    \item Prepared homeworks (using Jupyter notebooks for the first time) and led homework review
      sessions.
    \item Lectured in professor's absence.
    \item Received competitive departmental Teaching Assistant award.
  \end{rSubsection}
  \begin{rSubsection}{Undergraduate mentor}{2012 - 2013, 2015 - 2017}{6 semesters}{UW-Madison}
    \item Designed appropriate experiments that were complementary to my own research.
    \item Introduced undergraduates to spectroscopy, programming, and instrument design
  \end{rSubsection}
  \begin{rSubsection}{General Chemistry II}{2011, 2012}{Teaching Assistant, 2
      semesters}{UW-Madison}
    \item Coordinated two sections---total of $\approx50$ students in each semester.
    \item Led labs.
    \item Designed and led discussion sections.
  \end{rSubsection}
  \begin{rSubsection}{General Chemistry}{2010, 2011}{Peer Science Leader, 2 semesters}
    {Bates College}
    \item Designed and led class-wide review sessions for General Chemistry.
    \item Assisted in first trials of new peer leadership program at Bates College.
    \item Attended regular meetings to share teaching strategies with other peer leaders.
  \end{rSubsection}
\end{rSection}
\begin{rSection}{Skills \& Specialties}
  \begin{rSubsection}{Analytical Techniques}{}{}{}
    \item Spectroscopy: Raman / IR / UV-VIS / NMR
    \item Ultrafast Spectroscopy: Pump Probe / CMDS
    \item Optomechanical design and construction
  \end{rSubsection}
  \begin{rSubsection}{Computer Programs \& Programming Languages}{}{}{}
    \item Python (SciPy, PyQT4, PyPI, micropython)
    \item git
    \item KiCad
    \item Anaconda, conda-forge
    \item Arduino (Extended C++)
    \item LaTeX
    \item LabView
  \end{rSubsection}
\end{rSection}

\pagebreak

\begin{rSection}{Service Activites \& Community Involvement}
  \begin{rSubsection}{Plasma Group Python Introduction}{2017}{Assistant}{UW-Madison}
    \item Helped introduce a group of Faculty and Graduate Students in Physics to Python.
    \item Group was switching to Python from IDL.
    \item Introduction consisted of weekly meetings across several months.
  \end{rSubsection}
  \begin{rSubsection}{Pre-college Enrichment Opportunity Program for Learning Excellence (PEOPLE)}
    {2017}{Volunteer}{Madison WI}
	  \item Taught disadvantaged high school students about electronics, science and what it is like
      to be an analytical chemist.
  \end{rSubsection}
  \begin{rSubsection}{Wisconsin Middle School Science Bowl}{2017}{Scientific Judge}{Madison WI}
    \item Judged middle school students in statewide science-knowledge competition.
    \item Winning team proceeded to national competition.
  \end{rSubsection}
  \begin{rSubsection}{McElvain Committee}{2013 - 2014}{Committee Member}{UW-Madison}
    \item Graduate student committee to choose seminar speakers.
  \end{rSubsection}
  \begin{rSubsection}{Freewill Folk Society}{2008 - 2011}{President}{Bates College}
    \item Reorganized club structure, recruited other students to new club positions.
    \item Organized monthly folk dances, bringing in bands and callers.
  \end{rSubsection}
  \begin{rSubsection}{OutFront}{2009 - 2010}{President}{Bates College}
    \item Held weekly meetings, organized social and political events.
    \item For the first time, organized and executed collaborative events with other local LGBTQ
      groups in Lewiston ME.
  \end{rSubsection}
\end{rSection}

\end{document}